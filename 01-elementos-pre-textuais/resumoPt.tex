%
% Documento: Resumo (Português)
%

\begin{RESUMO}
\thispagestyle{empty}
	\begin{SingleSpace}


		Em muitas empresas existe uma carência quando o assunto é automação de testes ou processos web em navegadores. A necessidade de testes de regressão, testes de funcionalidades e ou
        automação de processos cresce junto com o sistema, porém a pratica dessas atividades só ganha força quando aparece alguma necessidade ou problema.

        Para resolver este problema, propõe-se neste trabalho, um \textit{framework} com o objetivo de trazer aos usurários uma ferramenta de fácil uso e com recursos úteis para o desenvolvimento dessas tarefas, contando com a facilidade e versatilidade da
        linguagem \textit{Python} e a integração de navegadores com o \textit{framework} \textit{Selenium}.

        O conjuntos de ferramentas que o \textit{framework} proposto apresenta são: gerenciamento automático dos controladores de navegadores(\textit{drivers}), módulo de relatórios e \textit{logs} para controle de atividades executadas,
        padronização de criação de elementos de tela utilizando o padrão PageObject e a identificação de alteração de \textit{layout}.

		\vspace*{0.5cm}\hspace{-1.3 cm}\textbf{Palavras-chave}: Automação, Selenium, Testes.

	\end{SingleSpace}
\end{RESUMO}


