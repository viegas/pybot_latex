%
% Documento: Resumo (Português)
%

\begin{RESUMO}
\thispagestyle{empty}
	\begin{SingleSpace}


		Em muitas empresas temos uma certa carência quando o assunto é automação de testes ou processos web em navegadores. A necessidade de testes de regressão, testes de funcionalidades e ou
        automação de processos cresce junto com o sistema, porem a pratica dessas atividades só tem força quando aparece aguma necessidade ou problema.

        Esse novo framework pretende trazer aos usuarios uma ferramenta de facil uso e com recursos uteis para o desenvolvimento dessas tarefas, contando com a facilidade e versatilidade da
        linguagem python e a integração com navegadores com framework selenium.

        O conjuntos de ferarmentas que o framework dispõe são: gerenciamento automatico dos controladores de navegadores(drivers), modulo de relatorios e logs para controle de atividades executadas,
        padronização de criação de elementos de tela utilizando o padrão PageObject e a identificação de alteração de layout.

		\vspace*{0.5cm}\hspace{-1.3 cm}\textbf{Palavras-chave}: Automação, Selenium, Testes.

	\end{SingleSpace}
\end{RESUMO}


