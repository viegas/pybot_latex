%
% Documento: Introdução
%

\chapter{INTRODUÇÃO}\label{chap:introducao}

    O ciclo de vida de software tem diversas etapas, das quais podem ser elencadas: Análise de requisitos, Concepção do Projeto, Desenvolvimento,
    Implantação e por fim Manutenção. Usualmente, nas etapas de Desenvolvimento e Manutenção ocorre a maior parte da criação ou a codificação do
    \textit{software}, e na concepção de um projeto a necessidade da criação de um processo de testes que cresça junto do sistema não costuma ter
    a relevância necessária.

    Atualmente, existem diversas ferramentas que possibilitam a criação de testes automatizados como por exemplo o \textit{Selenium Webdriver} \cite{selenium}
    e o \textit{Robotframework} \cite{robotframework}, e esses testes automatizados podem ser desde os mais simples, como a verificação de campos dentro da página
    ou um determinado título, até testes mais complicados, como um teste de regressão, onde todas as funcionalidades e requisitos
    do software são testadas novamente para garantir que uma atualização do software não impacte em outras partes. Porém, para fazer uso dessas
    ferramentas existentes é necessário avaliar previamente diferentes questões, como compatibilidade com sistema operacional, linguagem suportada,
    funcionalidades oferecidas e padrões a serem utilizados, e a curva de aprendizagem para começar a utilizá-las pode ser demorada e custosa.

    Este trabalho de conclusão de curso tem como objetivo criar um \textit{framework} para auxiliar nas tarefas codificação e criação de processos de testes automatização para
    sistemas e ou sites em navegadores, contando com uma forma de utilização fácil e simples, e trazendo para si algumas das preocupações básicas que os desenvolvedores enfrentam
    ao utilizar outros \textit{frameworks} de testes. Levando o nome de PyBot, união das palavras \textit{Python}, linguagem utilizada para criação do projeto, e \textit{Bot},
    que em inglês quer dizer robô, essa ferramenta propõe disponibilizar para os usuários uma série de ferramentas para auxiliar a criação e implantação desses processos,
    contando com uma estrutura de criação dos \textit{scripts} de testes. Para isso, esse \textit{framework} contará com uma instalação simples com o minimo de configurações e permitirá
    a criação de testes no padrão \textit{PageObject} e \textit{PageElement}, geração de registros de \textit{logs} para controle de tarefas e passos executados e o
    gerenciamento automático de \textit{drivers} de navegadores com a ferramenta \textit{Driloader}.

    Com isso a codificação de \textit{scripts} de testes automatizados pode se tornar mais simples, pois parte das complexidades de se iniciar com esses testes automatizado será
    feita inteiramente pelo \textit{framework}, encarregando o usuário em apenas na  codificação das páginas do seu site no modelo simples e intuitivo do \textit{Pybot} e na criação dos métodos que
    serão encarregados de executar os determinados comandos daquela página.

    O restante deste documento é organizado da seguinte forma:
    No \autoref{chap:relacionados} serão apresentados os trabalhos relacionado, contando com 2 exemplos de ferramentas existentes e um comparativo dessas ferramentas e o \textit{framework} proposto;
    No \autoref{chap:imp} irá apresentar os módulos presentes no \textit{framework} proposto com seus determinados propósitos e funcionalidades;
    No \autoref{chap:tec} irá apresentar as tecnologias utilizadas para o desenvolvimento e controle do código fonte do \textit{framework};
    No \autoref{chap:proj} irá apresentar um proposta de utilização do \textit{framework}, fazendo uso de alguns conceitos apresentados;
    No \autoref{chap:result} irá apresentar uma análise de dados coletados a partir de um questionário com usuários sobre a utilização do \textit{framework};
    E por fim no \autoref{chap:conclusao} será apresentado a conclusão deste trabalho junto de possíveis trabalhos futuros.