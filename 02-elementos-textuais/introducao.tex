%
% Documento: Introdução
%

\chapter{INTRODUÇÃO}\label{chap:introducao}

O ciclo de vida de software tem diversas etapas, das quais podem ser elencadas: Análise de requisitos, Concepção do Projeto, Desenvolvimento,
Implantação e por fim Manutenção. Usualmente, nas etapas de Desenvolvimento e Manutenção ocorre a maior parte da criação ou a codificação do
\textit{software}, e na concepção de um projeto a necessidade da criação de um processo de testes que cresça junto do sistema não costuma ter
a relevância necessária.

Atualmente, existem diversas ferramentas que possibilitam a criação de testes automatizados tais como \emph{Selenium Webdriver} \cite{selenium}
e o \emph{Robotframework} \cite{robotframework}, tais testes podem ser desde os mais simples, como a verificação de campos dentro da pagina
ou um determinado titulo, até testes mais complicados, como um teste de regressão, onde todas as funcionalidades e requisitos
do software são testadas novamente para garantir que uma atualização do software não impacte em outras partes. Porém, para fazer uso dessas
ferramentas existentes é necessário avaliar diferentes questões previamente, como compatibilidade com sistema operacional, linguagem suportada
e ferramentas oferecidas, e a curva de aprendizagem para começar a utiliza-las pode ser demorada e custosa.

Portanto, este Trabalho de Conclusão de Curso tem como objetivo criar um \textit{framework} para auxiliar nas tarefas de criação de testes e ou
automatização de processos de sistemas executados em navegadores, contando com uma forma de utilização fácil e trazendo para si algumas das
preocupações básicas que os desenvolvedores enfrentam ao utilizar outras ferramentas de testes. Levando o nome de PyBot, união das palavras \emph{Python},
linguagem utilizada para criação do projeto, e \emph{Bot}, que em inglês quer dizer robô, essa ferramenta propõe prover para os usurários uma série de
ferramentas para auxiliar a criação e implantação desses processos, contando com uma estrutura de criação dos \emph{scripts} de testes. Para isso,
a solução permitirá a criação de testes no padrão \emph{PageObject} e \emph{PageElement}, geração de registros de \emph{logs} para controle de tarefas e passos
executados e o gerenciamento automático de \emph{drivers} de navegadores com a ferramenta \emph{Driloader}.


O restante do Trabalho de Conclusão de Curso é organizado da seguinte forma:
O \autoref{chap:relacionados} serão apresentados os trabalhos relacionado, contando com 2 exemplos de ferramentas existentes e um comparativo dessas ferramentas e o framework proposto;
O \autoref{chap:imp} irá apresentar os módulos presentes framework com seu determinado propósito e funcionalidades;
O \autoref{chap:tec} irá apresentar as tecnologias utilizadas para o desenvolvimento e controle do código fonte do framework;
O \autoref{chap:proj} irá apresentar um proposta de utilização do framework, fazendo uso de alguns conceitos apresentados.
O \autoref{chap:result} irá apresentar uma análise de dados coletados a partir de um questionário com usuários sobre a utilização do framework.
E por fim no \autoref{chap:conclusao} será apresentado a conclusão deste trabalho junto de possíveis trabalhos futuros.