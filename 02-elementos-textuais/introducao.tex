%
% Documento: Introdução
%

\chapter{INTRODUÇÃO}\label{chap:introducao}

O ciclo de vida de software tem diveras etapas, de um modo geral elas são: Análise de requisitos, Concepção do Projeto, Desenvolvimento, Implantação e por fim Manutenção.
Nas etapas de Desenvolvimento e a Manutenção é onde a criação ou a codificação do software em questão mais acontece, e na concepção de um projeto a necessidade da criação
de um processo de testes que cresça junto do sistema não tem a sua devida importação.


Este Trabalho de Conclusão de Curso tem como objetivo criar um framework para auxiliar nas tarefas de criação de testes e ou automatização de processos de sistemas executados em navegadores.
Levando o nome de PyBot, união das palavras Python, liguagem utilizada para criação do projeto, e Bot, que em inglês quer dizer robô, essa ferramenta propõe prover para os usuarios uma serie
de ferramentas para auxiliar a criação e implantação desses processos, contando uma estrutura de criação dos scripts de teste no padrão PageObject e PageElement, geração de registros de logs
para controle de tarefas e passos executados, verificação de alteração de interfaces e layout dos sites e o gerenciamento automatico de drivers de navegadores com a ferramenta Driloader.