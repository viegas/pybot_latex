%
% Documento: Disposições
%

\chapter{TECNOLOGIAS EMPREGADAS}\label{chap:tec}
        Para o desenvolvimento do \textit{framework} foi utilizado apenas como linguagem para desenvolvimento o \emph{Python}
        e para a manipulação e integração com o navegador a biblioteca em \emph{Python} do Selenium \emph{Webdriver}. Por ser um
        projeto que visa ser o mais simples e leve possível apenas os módulos padrões do \emph{Python} estão sendo utilizado
        para o desenvolvimento desta ferramenta.


        \section{Python}

            A escolha do \emph{Python} \cite{Python} foi devida porque ele trata-se de uma linguagem de programação fácil de aprender e poderosa.
            Possuindo uma estruturas dados de alto nível e uma abordagem simples, mas eficaz, para a programação orientada
            a objetos. Contendo uma Sintaxe elegante e tipagem dinâmica, juntamente com uma interpretação natural, tornam
            a linguagem ideal para criação dos \emph{scripts} do Pybot.

        \section{Selenium WebDriver}
        \label{webdriver}
            \emph{Selenium Webdriver} \cite{webdriver} é um \emph{framework} utilizado para se comunicar e enviar comandos para os navegadores
            em conjunto com um controlador de cada navegador específico. Em comparação com seu antecessor, \emph{Selenium RC}, o \emph{Selenium Webdriver}
            não precisa de um \emph{server} para enviar os comandos para o navegador. Utilizando comando nativos do sistema operacional ao invés de
            comando \emph{javascript}, usados pelo \emph{Selenium RC}, deixam o \emph{Selenium Webdriver} uma excelente ferramenta para integração com diversos navegadores.

        \section{Git e GitHub}
            Para o controle de versões e alterações do código fonte do \emph{framework} e \emph{scripts} de exemplo foi utilizado a ferramenta
            \emph{Git} \cite{git} em conjunto com os servidores do \emph{Github} \cite{github} para hospedagem e gerenciamento. Com eles foi possível
            fazer alterações dos códigos fontes em qualquer computador e gerenciar os erros e melhorias do \emph{framework}.
