%
% Documento: Introdução
%

\chapter{TRABALHOS RELACIONADOS}\label{chap:relacionados}

    Conforme \cite{dos2016estudo} a importância da criação de processos automatizados cresce junto da importância que os \emph{softwares} vão tendo na sociedade, e as empresas
    tem uma certa dificuldade quando tentam de adotar ou implantar tais tipos de processos, devido a falta de profissionais com esse conhecimento ou pelas técnicas presentes hoje.

    A solução para automação de testes e processos em navegadores com maior adoção pela comunidade de desenvolvimento de software é o \emph{Selenium} \cite{selenium}.
    O \emph{Selenium} trata-se de uma ferramenta composto por diversos projetos tais como \emph{Selenium Grid}, \emph{Selenium IDE}, \emph{Selenium Remote Control} e o \emph{Selenium WebDriver},
    cada um com suas determinadas funcionalidades. Dentre os projetos citados, o \emph{Selenium WebDriver} é o que mais se assemelha ao projeto proposto pois trata-se de um \emph{framework} para
    integração com navegador, este por sua vez é utilizado como uma das dependências do Pybot(\ref{webdriver}).

    Ainda, outra ferramenta também relacionada ao projeto proposto é o \emph{Robotframework}, esta por sua vez é uma solução bem mais sofisticada, contendo uma quantidade abrangente de módulos
    e usos, porém junto traz uma complexidade maior para seu uso.

    Similar a estes sistemas, a solução proposta neste Trabalho de Conclusão de Curso busca seguir direcionando a integração do \emph{Selenium} com o \emph{Python}. Porém, pretende-se com o Pybot o
    desenvolvimento ferramenta simples e de fácil uso, portanto.

    Para entender melhor o problema, apresenta-se uma análise das funcionalidades e usos das ferramentas citadas anteriormente junto de um comparativo de alguns pontos fortes e
    fracos de cada uma delas.

    \section{Selenium Webdriver}

        \emph{Selenium Webdriver} \cite{webdriver} trata-se de um \emph{framework} onde disponibiliza-se para usuário uma API para integração com o navegador escolhido. Com essa API é possível enviar
        diversos tipos de comandos para o navegador, tais como, verificação e iteração qualquer tipo de elemento presente no DOM (Modelo de Documento Objeto, do inglês \emph{Document Object Model}),
        geração de \emph{prints} de tela e manipulação do próprio navegador, como por exemplo maximizar a janela, redimensionar ou abrir uma nova janela. Todos os comando executados no navegador
        são comandos nativos de cada sistema operacional, sendo necessário possuir o \emph{driver} especifico para cada navegador e sistema operacional para que funcione.

        Ainda, possui suporte as seguintes linguagens de programação: \emph{Java}, \emph{Csharp}, \emph{Python}, \emph{Ruby}, \emph{Php}, \emph{Perl} e \emph{Javascript}. Na maioria
        dos casos, o suporte e funcionalidade são os mesmos para todas linguagens, porém o maior suporte é dado para a linguagem \emph{Java}.


    \section{Robotframework}

        {Robotframework \cite{robotframework} é um dos \emph{frameworks} mais abrangentes disponíveis para teste de software para linguagem \emph{Python}. Esta solução consiste
        de uma vasta variedade de módulos distintos, contando com 11 módulos próprio do \emph{framework} e mais 30 de projetos parceiros, possíveis de serem habilitados. Desta forma,
        possível de optar por incluir determinado módulo ou não no projeto, tendo, também, suporte a \emph{Java} na maioria de seus módulos.

        Sua arquitetura foi feita para executar testes utilizando-se de ATDD (\emph{Acceptance Test Driven Development} ou Desenvolvimento Orientado a Testes de Aceitação) que trata-se de uma
        abordagem ou prática para a criação de requisitos colaborativamente entre o cliente e a equipe e fazendo uso da técnica de desenvolvimento Ágil BDD
        (\emph{Behavior Driven Development} ou Desenvolvimento Guiado por Comportamento em português) onde são criados cenários para cada tipo de comportamento da aplicação é criado levando em
        conta 3 estados, Dado que, Quando e Então, como no exemplo a seguir.

        \vspace*{0,5cm}
        Exemplo: Listas com alguma coisa dentro não podem estar vazias
        \begin{description}
            \item[Dado] que uma nova lista é criada
            \item[Quando] eu adiciono um objeto a ela
            \item[Então] a lista não deve estar vazia.
        \end{description}
        \vspace*{0,5cm}

        O \emph{Robotframework} conta também com os seguinte recursos:
        \begin{itemize}
            \item geração de relatórios de execução, tanto de sucesso como em falhas;
            \item interface por linha de comando;
            \item resposta de execução por XML.
        \end{itemize}


    \section{Comparativo}

    Foram levantados alguns dos pontos mais relevantes para a execução dos \emph{scripts} de testes de cada um dos \emph{framework} e feito uma tabela comparativa entre eles.
    Ambos \emph{frameworks} utilizam em sua base o próprio \emph{framework} \emph{Selenium Webdriver}, portando todas as funcionalidades de integração com os navegadores estarão
    presentes neles e não serão tratadas no comparativo feito a seguir na \autoref{tab:comp}.

    \vspace*{1cm}
\begin{table}[H]
\centering
\caption{Comparativo dos Frameworks}
\label{tab:comp}
\setlength\extrarowheight{10pt}
\begin{tabular}{l|c|c|c}
                                                                                    & Selenium                                                                                  & Robotframework & Trabalho Proposto \\ \hline
Integração com navegador                                                            & Sim                                                                                       & Sim            & Sim               \\ \hline
\begin{tabular}[c]{@{}l@{}}Gerenciamento de drivers\\ dos navegadores\end{tabular}  & Não                                                                                       & Não            & Sim               \\ \hline
Linguagens Suportadas                                                               & \begin{tabular}[c]{@{}c@{}}Java, C\#, Python, Ruby,\\ Php, Perl e Javascript\end{tabular} & Python e Java  & Python            \\ \hline
\begin{tabular}[c]{@{}l@{}}Supote ao conceito \\ de Page Object\end{tabular}        & Não                                                                                       & Sim            & Sim               \\ \hline
\begin{tabular}[c]{@{}l@{}}Suporte à Técnicas\\ de Desenvolvimento\end{tabular}     & Nenhuma                                                                                   & ATDD e BDD     & Nenhuma           \\ \hline
Relatorios de Execução                                                              & Nenhum                                                                                    & Sucesso/Erro   & Sucesso/Erro      \\ \hline
\begin{tabular}[c]{@{}l@{}}Verificação dinâmica \\ dos elementos \end{tabular}      & Não                                                                                       & Sim            & Sim               \\ \hline
Arquivo de Configuração                                                             & Não                                                                                       & Sim            & Sim               \\ \hline
Modular                                                                             & Não                                                                                       & Sim            & Não               \\ \hline
Código Aberto                                                                       & Sim                                                                                       & Sim            & Sim
\end{tabular}
\end{table}
\vspace*{-0,9cm}



    Como podemos ver, o diferencial é nas funcionalidades especificas de cada um, onde podemos perceber uma separação de níveis. Onde temos uma ferramenta super completa com módulos independentes
    e específicos para cada necessidade dando uma versatilidade maior para usuário em questão de funcionalidades, que é o caso do \emph{Robotframework}, e por outro lado temos outros 2 \emph{frameworks}
    mais básicos, onde o \emph{Pybot} traz melhoria em relação ao \emph{Selenium} padrão, dando uma maior facilidade para os usuários no inicio das tarefas utilizando algumas técnicas para agilizar e organizar
    o desenvolvimento dos \emph{scripts}, relatórios de execução, arquivo de configurações para execução dos \emph{scripts} e controle automático dos \emph{drivers} de cada navegador.


