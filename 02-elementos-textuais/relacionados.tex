%
% Documento: Introdução
%

\chapter{TRABALHOS RELACIONADOS}\label{chap:relacionados}

    Conforme Max dos Santos \cite{dos2016estudo} a importância da criação de processos automatizados cresce junto da importância que os \textit{softwares}
    vão tendo na sociedade, e as empresas tem uma certa dificuldade quando tentam de adotar ou implantar tais tipos de processos, devido à falta de
    profissionais com esse conhecimento ou pelas técnicas presentes hoje.

    A solução para automação de testes e processos em navegadores com maior adoção pela comunidade de desenvolvimento de software é o \textit{Selenium}
    \cite{selenium}. O \textit{Selenium} trata-se de uma ferramenta composto por diversos projetos tais como \textit{Selenium Grid}, \textit{Selenium IDE},
    \textit{Selenium Remote Control} e o \textit{Selenium WebDriver}, cada um com suas determinadas funcionalidades. Dentre os projetos citados, o
    \textit{Selenium WebDriver} é o que mais se assemelha ao projeto proposto pois trata-se de um \textit{framework} para integração com navegador,
    este por sua vez é utilizado como uma das dependências do Pybot (\autoref{webdriver}).

    Ainda, outra ferramenta também relacionada ao projeto proposto é o \textit{Robotframework}, que se trata de uma solução bem mais sofisticada,
    contendo uma quantidade abrangente de módulos e usos, porém junto traz uma complexidade maior para seu uso.

    Similar a estes sistemas, a solução proposta neste Trabalho de Conclusão de Curso busca seguir direcionando a integração do \textit{Selenium} com
    o \textit{Python}. Porém, pretende-se com o Pybot o desenvolvimento ferramenta simples e de fácil uso, portanto.

    Para entender melhor o problema, apresenta-se uma análise das funcionalidades e usos das ferramentas citadas anteriormente junto de um comparativo
    de alguns pontos fortes e fracos de cada uma delas.

    \section{Selenium Webdriver}
    \label{sec:selenium}
        O \textit{Selenium Webdriver} \cite{webdriver} é de um \textit{framework} onde disponibiliza-se para usuário uma \textit{API}, Interface de
        Programação de Aplicação (do Inglês Application Programming Interface - API), para integração com o navegador escolhido. Com essa \textit{API}
        é possível enviar diversos tipos de comandos para o navegador, tais como, verificação e iteração qualquer tipo de elemento presente no DOM
        (Modelo de Documento Objeto, do inglês \textit{Document Object Model}), geração de \textit{prints} de tela e manipulação do próprio navegador,
        como por exemplo maximizar a janela, redimensionar ou abrir uma nova janela. Todos os comandos executados no navegador são comandos nativos de
        cada sistema operacional, sendo necessário possuir o \textit{driver} especifico para cada navegador e sistema operacional para que funcione.

        Ainda, possui suporte as seguintes linguagens de programação: \textit{Java}, \textit{Csharp}, \textit{Python}, \textit{Ruby}, \textit{Php},
        \textit{Perl} e \textit{Javascript}. Na maioria dos casos, o suporte e funcionalidade são os mesmos para todas linguagens, porém o maior suporte
        é dado para a linguagem \textit{Java}.


    \section{Robotframework}

        O \textit{Robotframework} \cite{robotframework} é um dos \textit{frameworks} mais abrangentes disponíveis para teste de software para linguagem
        \textit{Python}. Esta solução consiste de uma vasta variedade de módulos distintos, contando com 11 módulos próprio do \textit{framework} e
        mais 30 de projetos parceiros, possíveis de serem habilitados. Desta forma, possível de optar por incluir determinado módulo ou não no projeto,
        tendo, também, suporte a \textit{Java} na maioria de seus módulos.

        Sua arquitetura foi feita para executar testes utilizando-se de ATDD (\textit{Acceptance Test Driven Development} ou Desenvolvimento Orientado
        a Testes de Aceitação) que trata-se de uma abordagem ou prática para a criação de requisitos colaborativamente entre o cliente e a equipe e
        fazendo uso da técnica de desenvolvimento Ágil BDD (\textit{Behavior Driven Development} ou Desenvolvimento Guiado por Comportamento em
        português) onde são criados cenários para cada tipo de comportamento da aplicação levando em conta 3 estados, \textbf{Dado que}, \textbf{Quando}
        e \textbf{Então}, como no exemplo a seguir. Como por exemplo um caso de teste onde temos que verificar uma lista com objetos dentro não podem
        estar vazias, a escrita desse teste ficaria da seguinte maneira: \textbf{Dado} que uma nova lista é criada; \textbf{quando} eu adiciono um objeto
        a ela; \textbf{então} a lista não deve estar vazia. Dessa maneira para o \textit{Robotframework} cara um desses estados é um passo a ser executado,
        e o mesmo foi codificado para se adequar a esse caso de teste O \textit{Robotframework} conta também com os seguintes recursos:

        \begin{itemize}
            \item Geração de relatórios de execução, tanto de sucesso como em falhas;
            \item Interface por linha de comando;
            \item Resposta de execuções por XML.
        \end{itemize}


    \section{Comparativo}

        Foram levantados alguns dos pontos mais relevantes para a execução dos \textit{scripts} de testes de cada um dos \textit{frameworks} e feito uma
        tabela comparativa entre eles. Todos os \textit{frameworks} apresentados no comparativo a seguir na \autoref{tab:comp} utilizam em sua base o
        próprio \textit{Selenium Webdriver}, portanto todas as funcionalidades padrões de integração com os navegadores descritas na \autoref{sec:selenium}
        não serão tratadas nesse comparativo.

        \vspace*{1cm}
\begin{table}[H]
\centering
\caption{Comparativo dos Frameworks}
\label{tab:comp}
\setlength\extrarowheight{7pt}
\begin{tabular}{l|c|c|c}
                                                                                & Selenium                                                                                  & Pybot   & Robotframework \\ \hline
Integração com browser                                                          & Sim                                                                                       & Sim     & Sim            \\ \hline
\begin{tabular}[c]{@{}l@{}}Gerenciamento de drivers\\ dos browser\end{tabular}  & Não                                                                                       & Sim     & Não            \\ \hline
Linguagens Suportadas                                                           & \begin{tabular}[c]{@{}c@{}}Java, C\#, Python, Ruby,\\ Php, Perl e Javascript\end{tabular} & Python  & Python e Java  \\ \hline
\begin{tabular}[c]{@{}l@{}}Supote ao conceito \\ de Page Object\end{tabular}    & Não                                                                                       & Sim     & Sim            \\ \hline
\begin{tabular}[c]{@{}l@{}}Suporte à Técnicas\\ de Desenvolvimento\end{tabular} & Nenhuma                                                                                   & Nenhuma & TDD,BDD e DDT  \\ \hline
Relatorios de Execução                                                          & Nenhum                                                                                    & Erro    & Sucesso e Erro \\ \hline
Arquivo de Configuração                                                         & Não                                                                                       & Sim     & Sim            \\ \hline
Modular                                                                         & Não                                                                                       & Não     & Sim            \\ \hline
Código Aberto                                                                   & Sim                                                                                       & Sim     & Sim
\end{tabular}
\end{table}

\vspace*{-0,9cm}
{\raggedright \fonte{Felipe Viegas, 2017.}}
\vspace*{1cm}


        \vspace*{1,7cm}

        Como podemos ver na \autoref{tab:comp}, o diferencial é nas funcionalidades especificas de cada um dos \textit{frameworks}, onde podemos perceber
        uma separação de níveis. Onde o \textit{Robotframework} uma ferramenta completa, contem módulos independentes e específicos para cada necessidade,
        dando assim uma versatilidade maior para usuário e o \textit{Selenium Webdriver} sendo mais básico, onde não possui muito alem de prover a
        integração com o navegador e suporte a diversas linguagens de programação.

        O projeto proposto pretende trazer melhorias em relação ao \textit{Selenium Webdriver} padrão, mas não pretende ser uma ferramenta tão abrangente
        quanto o \textit{Robotframework}, sendo um meio termo entre essas ferramentas. Para isso o projeto propõe dar uma maior facilidade para os usuários
        no início das tarefas, gerenciando automaticamente os \textit{driver} dos navegadores, utilizando-se de algumas técnicas para agilizar e organizar
        o desenvolvimento dos \textit{scripts} com o padrão \textit{Page Object} e a pesquisa dinâmica dos elementos nas páginas, geração de relatórios de
        execução e contendo um arquivo de configurações centralizado para as execuções dos \textit{scripts}.
