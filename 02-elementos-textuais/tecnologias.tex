%
% Documento: Disposições
%

\chapter{TECNOLOGIAS EMPREGADAS}\label{chap:tec}

        Para o desenvolvimento do \textit{framework} foi utilizado apenas como linguagem para desenvolvimento o \textit{Python} na versão \textit{3.6}
        e para a manipulação e integração com o navegador a biblioteca em \textit{Python} do \textit{Selenium Webdriver} tambem na versão \textit{3.6}.
        Por ser um projeto que visa ser o mais simples e leve suas dependências são minimas, fazendo uso apenas dos módulos padrões do \textit{Python}
        e do \textit{Selenium Webdriver} para o desenvolvimento desta ferramenta.

        \section{Python}

            Foi escolhido o \textit{Python} \cite{Python} por que se trata de uma linguagem de programação fácil de aprender e poderosa. Possuindo uma
            estrutura de dados de alto nível e uma abordagem simples, mas eficaz, para a programação orientada a objetos. Contendo uma Sintaxe elegante
            e tipagem dinâmica, juntamente com uma interpretação natural, tornam a linguagem ideal para criação dos \textit{scripts} do Pybot.

        \section{Selenium WebDriver}
        \label{webdriver}

            \textit{Selenium Webdriver} \cite{webdriver} é um \textit{framework} utilizado para se comunicar e enviar comandos para os navegadores em
            conjunto com um controlador específico de cada navegador. Em comparação com seu antecessor, \textit{Selenium RC}, o \textit{Selenium Webdriver}
            não precisa de um \textit{server} para enviar os comandos para o navegador, sendo preciso apenas a utilização desse controlador. Utilizando
            comando nativos do sistema operacional ao invés de comando \textit{javascript}, usados pelo \textit{Selenium RC}, deixam o \textit{Selenium Webdriver}
            uma excelente ferramenta para integração com diversos navegadores.

        \section{Git e GitHub}

            Para o controle de versões e alterações do código fonte do \textit{framework} e \textit{scripts} de exemplo foi utilizado a ferramenta
            \textit{Git} \cite{git} em conjunto com os servidores do \textit{Github} \cite{github} para hospedagem e gerenciamento. Com eles foi possível
            fazer alterações dos códigos fontes em qualquer computador e gerenciar os erros e melhorias do \textit{framework}.
