%
% Conclusão
%

\chapter{CONCLUSÃO}\label{chap:conclusao}


    Este trabalho apresentou um \textit{framework} para automatização de testes e processos em navegadores utilizando \textit{Python} e o
    \textit{framework} \textit{Selenium Webdriver} como base, sendo assim possível a criação de \textit{scripts} de testes automatizado em qualquer
    tipo de ambiente sem a necessidade de configurações e instalações complexas, contando com ferramentas que facilitam e agilizam o processo de
    criação desses \textit{scripts} com padrões de \textit{Page Object}.

    Com base nos resultados obtidos pelo questionário aplicado aos usuários no \autoref{chap:result} pode-se observar que o \textit{framework} tem
    potencial para auxiliar os desenvolvedores a criar processos de testes com uma certa facilidade em relação aos \textit{frameworks} existentes
    atualmente, conforme o objetivo do projeto.

    Por fim, com base no desenvolvimento e andamento do projeto e juntos das análises de dados foi notado algumas melhorias e trabalhos futuros faltaram
    para tornar o \textit{Pybot} numa ferramenta mais completa, como a criação de uma documentação completa dos módulos, controle de processos assíncronos
    do \textit{Javascripts}, gerenciamento de múltiplas janelas e abas e a hospedagem do \textit{framework} no repositório padrão do \textit{Python}.
