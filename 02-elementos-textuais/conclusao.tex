%
% Conclusão
%

\chapter{CONCLUSÃO}\label{chap:conclusao}


    Este trabalho apresentou um \emph{framework} para automatização de testes e processos em navegadores utilizando \emph{Python} e o \emph{framework} \emph{Selenium Webdriver}
    como base, sendo possível a criação de \emph{scripts} de testes em qualquer ambiente sem a necessidade de configurações e instalações complexas.

    Com base nos resultados obtidos pelo questionário aplicado ao usuários no \autoref{chap:result} pode-se observar que o \emph{framework} tem potencial
    para auxiliar os desenvolvedores a criar processos de testes com uma certa facilidade em relação ao \emph{frameworks} existentes atualmente,
    conforme o objetivo do projeto.

    Por fim, com base no desenvolvimento e andamento do projeto foi notado algumas melhorias e trabalhos futuros faltaram para tornar o \emph{Pybot} numa ferramenta
    mais completa, como a criação de uma documentação completa dos módulos, controle de processos assíncronos do \emph{Javascripts}, gerenciamento de múltiplas
    janelas e abas e a hospedagem do \emph{framework} no repositório padrão do \emph{Python}.
